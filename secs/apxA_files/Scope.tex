\section{Research Objectives and Scope}
\subsection{Approach to the research problem}
Our approach is based on the presumption that the aging detection problem can be framed as a state/parameter estimation challenge with respect to the concentration dynamics of the gases involved. This gives rise to the following sub-problems:

\begin{enumerate}
\item Determining a suitable model for the system dynamics.
\item Assessing whether the available data contains sufficient information for identifying the model parameters related to the chosen system dynamics.
\item Investigating the relationship between the parameters/states and the aging factor of the catalyst.
\item Understanding the uncertainties inherent in the aforementioned processes.
\item Finally, developing and validating the aging detection algorithm.
\end{enumerate}

\subsection{Objectives for 2023 (previously)}
\begin{enumerate}
 \item In the current data, the operating conditions in the truck cover a wider range than the ones in test-cell. So, we wish to validate the SCR-ASC model on test-cell data with operating conditions similar to the truck data. If it is difficult to obtain such experimental data in test-cell, then use a high-fidelity SCR-ASC model in AVL Boost, to validate our low fidelity diagnostics-oriented model across a wide range of transient and steady operating conditions.
 \item Explore other low fidelity diagnostics-oriented SCR-ASC model structures, that could be calibrated using the signals between SCR and ASC that could be obtained from the high-fidelity Boost model.
 \item Validate the model-based and data-driven OBD results using truck data from trucks with DG and/or an EUL catalyst.
 \item Validate the model-based and data-driven OBD results using high-fidelity simulations.
 \item Based on the model and OBD validation results, iterate and improve both model-based and data-driven approaches to achieve robust diagnostics performance validated using high-fidelity simulations and actual data. Also implement a more rigorous and detailed version of the stochastic model-based OBD.
 \item Support TTU in development and demonstration of model-based, data-driven, and frequency-domain OBD methods that rely on detecting NOx-rich regions where ASC is inactive.
\end{enumerate}


\subsection{Objectives for 2024}
\begin{enumerate}
\item Identify the model parameters and parametric uncertainties using
the test-cell data and data from AVL boost model and validate them.
\item Develop the estimation methodology for the storage capacity (Observer Design).
\item Relate catalyst's aging to the storage capacity and develop robust
non-intrusive aging detection algorithm and test the algorithm on the test-cell
data and the road data segments.
\end{enumerate}
\subsection{Objectives post-2024}
\begin{enumerate}
    \item Assess the potential to embed the algorithms for real-time computation.
    \item Implementation and comprehensive evaluation, including robustness analysis, of the proposed diagnostic algorithms in engine test cell and on the road.
\end{enumerate}


\subsection{Other requirements}
\begin{itemize}
    \item Prior to the end of this project, a proposal for follow-on work will be submitted to Cummins.
    \item Cummins to provide expertise via mentorship and regular feedback.
    \item Cummins to provide funds to support the students.
    \item Cummins to provide test-cell data with operating conditions similar to on-road trucks
    \item Cummins to provide truck data with at least one and preferably two known aging levels (could be DG and/or EUL)
    \item Cummins to provide access to high-fidelity AVL Boost model for SCR-ASC. This could be a black-box model with access to pre-SCR, post-SCR, post-ASC signals, and NH3 storage values in SCR and ASC.
\end{itemize}
