\subsection{Parametric state space models}

\subsubsection{Non-linear Parametric Model}
% ==============================================================================
The actual input to the system is urea [from say, AdBlue ($32.5\%$ aqueous urea
solution)] injection that converted to ammonia. This can be modelled by the following equation \cite{nova2014urea}:
\begin{align*}
    \dot C_{NH_3, in} &= - \frac{1}{\tau} C_{NH_3, in} + 2 \frac{1}{\tau} \frac{ \eta u_{inj}}{N_{urea} F}\\
    \text{where, } \quad &\\
    \tau &- \text{Time constant}\\
    u_{inj} &- \text{Mass injection rate of the AdBlue solution}\\
    \eta &- \text{Mass fraction of urea in the solution}\\
    N_{urea} &- \text{Atomic number of urea}\\
    F &- \text{Exhaust flow rate of the catalyst } m^3/s
\end{align*}

\itbf{Assumptions}:
\begin{enumerate}
    \item The above model assumes that the evaporation of the urea-solutions is a significantly slower process as
        compared to it's decomposition into ammonia. Thus, the reaction
        kinetics are neglected, and the evaporation is considered are a first
        order process w.r.t the vapor pressure of the ammonia.
    \item The injection dynamics are completely decoupled from that of other
        states.
    \item Further, it is observed that Urea is completely converted to Ammonia
        at the very upstream part of the SCR catalyst
        \cite{hsieh2011development}.
\end{enumerate}

Reparametrizing the above equation, let,
\begin{align*}
    x_4 &= C_{NH_3, in} \qquad b_u = 2 \frac{ \eta}{N_{urea}} \qquad \omega_u = \frac{1}{\tau}\\
    u_2 &= u_{inj}
\end{align*}

\begin{equation}{\label{eqn::urea_inj}}
    \dot x_4 = - \omega_u x_4 +   \frac{\omega_u b_u}{F} u_{2}
\end{equation}

Using the parameetric model with $x_3 = M_{NH_3}$ and introducing urea dosing
dyanmics from eqn.~\ref{eqn::urea_inj}. ($u_2$ becomes $x_4$.)
\begin{align*}
    \bm{x_1 \\ x_2 \\ x_3 \\ x_4} = \bm{C_{NO} \\ C_{NH_3} \\ M_{NH_3} \\ C_{NH_3, in}} \qquad &
    \bm{u_1 \\ u_2 } = \bm{C_{NO, in} \\ u_{inj}}
\end{align*}
\begin{align*}
    \mat{
    \\f_{13} &=& k_1
    \\f_{23} &=& k_{4F}
    \\f_{32} &=& k_{4F} \Theta
    \\f_{31} &=& k_1 V
    \\f_{24} &=& b_v F
    }
    \qquad
    \mat{
    \\ g_1    &=& b_v F
    \\ g_2    &=& b_v F + k_{4F} \Theta
    \\ g_{3}  &=& k_{4R}+k_3
    \\ g_{23} &=& k_{4R} V^{-1}
    \\ g_{32} &=& k_{4F} V \Theta
    \\ g_4 &=& \omega_u
    }
    \qquad
    \mat{
        b_{11} &=& b_v F
        \\
        b_{42} &=& \frac{\omega_u b_u}{F}
    }
\end{align*}
\begin{equation}{\label{eqn::full_nonlinear}}
     \bm{\dot x_1 \\
        \dot x_2\\
        \dot x_3\\
        \dot x_4} =
    \bm{
        -f_{13} x_1 x_3
        -g_1 x_1
        \\
        %===
        -g_2 x_2
        + f_{23} x_2 x_3
        + g_{23} x_3
        + f_{24} x_4
        \\
        %===
        -f_{32} x_2 x_3
        -g_3 x_3
        -f_{31} x_1 x_3
        + g_{32} x_2
        \\
        %===
        -g_4 x_4
    }
    + \bm{b_{11} & 0\\
          0     & 0\\
          0     & 0\\
          0     & b_{42}  }\bm{u_1 \\ u_2 }
\end{equation}

%===============================================================================
\subsubsection{Linearized Model}
The perturbation model would bring down the change in
temperature from exponent to the algebriac form. We have the first-order tayler
approximation of rate constant:
\begin{align*}
    k(\bar T + \delta T) &\approx k(\bar T) + \frac{E_a}{R\bar T^2} k(\bar T) \delta T\\
    \delta k &\approx p \bar k \delta T\\
    \text{Where,} \quad &\\
    p &= \frac{E_a}{RT_0^2}
\end{align*}
Introducing the above approximation into the lumped parameters from eqn.\ref{eqn::full_nonlinear}:
\begin{align*}
    f_{13} &= k_1 \quad&
    \implies \bar f_{13} &= \bar k_{1}, \quad&
    \delta f_{13} \delta T &= \lr{ p_{1} \bar k_{1} } \delta T\\
    %===
    f_{23} &= k_{4F} \quad &
    \implies \bar f_{23} &= \bar k_{4F}, \quad &
    \delta f_{23} \delta &= \lr{ p_{4F} \bar k_{4F} } \delta T\\
    %===
    f_{32} &= k_{4F} \Theta \quad &
    \implies \bar f_{32} &= \bar k_{4F} \Theta, \quad &
    \delta f_{32} \delta T &= \lr{ p_{4F}  \bar k_{4F} \Theta } \delta T
    \\
    f_{31} &= k_1 V \quad &
    \implies \bar f_{31} &= \bar k_1 V, \quad &
    \delta f_{31} \delta T &= \lr{ p_1 \bar k_1 V } \delta T
    \\
    f_{24} &= b_v F
    \quad &
    \implies \bar f_{24} &= b_v \bar F,
    \quad &
    \delta f_{24} \delta F &= b_v \delta F
    \\
    g_1    &= b_v F
    \quad &
    \implies \bar g_1 &= b_v \bar F,
    \quad &
    \delta g_1 \delta F &= b_v \delta F
    \\
    g_2    &= b_v F + k_{4F} \Theta
    \quad &
    \implies \bar g_2 &= b_v \bar F + \bar k_{4F} \Theta,
    \quad &
    \delta g_{2_F} \delta F + \delta g_{2_T} \delta T &= b_v \delta F +  \lr{ p_{4F}  \bar k_{4F}  \Theta } \delta T
    \\
    g_{3}  &= k_{4R}+k_3
    \quad &
    \implies \bar g_3 &= \bar k_{4R} + \bar k_3,
    \quad &
    \delta g_3 \delta T&= \lr{ p_{4R} \bar k_{4R} + p_3 \bar k_3} \delta T
    \\
    g_{23} &= k_{4R} b_v
    \quad &
    \implies \bar g_{23} &= \bar k_{4R} b_v,
    \quad &
    \delta g_{23} \delta T &= \lr{ p_{4R} \bar k_{4R} b_v } \delta T
    \\
    g_{32} &= k_{4F} V \Theta
    \quad &
    \implies \bar g_{32} &= \bar k_{4F} V \Theta,
    \quad &
    \delta g_{32} \delta T &= \lr{ p_{4F} \bar k_{4F} V \Theta } \delta T
    \\
    g_4 &= \omega_u
    \\
    b_{11} &= b_v F
    \quad &
    \implies \bar b_{11} &= b_v \bar F,
    \quad &
    \delta b_{11} \delta F &= b_v \delta F
    \\
    b_{42} &= \frac{\omega_u b_u}{F}
    \quad &
    \implies \bar b_{42} &= \frac{\omega_u b_u}{\bar F},
    \quad &
    \delta b_{42} \delta F &= -\lr{\frac{\omega_u b_u}{\bar F^2}} \delta F
\end{align*}

We have the linearized model of the system:

\begin{multline}\label{eqn::lin_model}
    \bm{\delta \dot x_1 \\ \delta \dot x_2 \\ \delta \dot x_3 \\ \delta \dot x_4} = \bm{-\lr{ \bar f_{13} \bar x_3 + \bar g_1 } &
                    0 &
                    -\bar f_{13} \bar x_1 &
                    0 \\
                    %===
                    0 &
                    \lr{ -\bar g_2 + \bar f_{23} \bar x_3 }&
                    \lr{\bar f_{23} \bar x_2 + \bar g_{23}}&
                    \bar f_{24} \\
                    %===
                    -\bar f_{31} \bar x_3 &
                    \lr{-\bar f_{32} \bar x_3 + \bar g_{32} } &
                    \lr{-\bar f_{32} \bar x_2 - \bar g_3 - \bar f_{31} \bar x_1 } &
                    0 \\
                    %===
                    0 & 0 & 0 & -g_4
                    }
    \bm{\delta x_1 \\ \delta x_2 \\ \delta x_3 \\ \delta x_4}\\
            +\bm{ \bar b_{11} &
                            0 &
                            \delta f_{13} \bar x_1 \bar x_3 &
                            \lr{\delta b_{11} \bar u_1- \delta g_1 \bar x_1}
                            \\
                        %===
                        0&
                        0 &
                        \lr{-\delta g_{2_T} \bar x_2 + \delta f_{23} \bar x_2 \bar x_3 + \delta g_{23} \bar x_3} &
                        \lr{-\delta g_{2_F} \bar x_2 + \delta f_{24} \bar x_4 }
                        \\
                        %===
                        0&
                        0&
                        \lr{-\delta f_{32} \bar x_2 \bar x_3 - \delta g_3 \bar x_3 - \delta f_{31} \bar x_1 \bar x_3 + \delta g_{32} \bar x_2 } &
                        0
                        \\
                        %===
                        0&
                        \bar b_{42}&
                        0&
                        \delta b_{42} \bar u_2
                        }
    \bm{\delta u_1 \\ \delta u_2 \\ \delta T \\ \delta F}
\end{multline}

\bigskip

\noindent\itbf{Remarks on the linearized model}\\

Following are a few remarks from the structure of $A$, $B$ matrices in the
linearized model:

\begin{enumerate}
    \item From the structure of $A$ matrix:
    \begin{enumerate}
\item Change in $NO_x$ concentration $(\delta x_1)$ will only influence the
catalyst storage dynamics of ammonia $(\delta x_3)$ other than itself.

\item Urea injection $(\delta x_4)$ will only influence the ammonia
concentration dynamics $(\delta x_2)$ other than itself.

\item Urea injection dynamics are only effected by its concentration and
no other states.

\item Ammonia storage and ammonia concentration are strongly coupled.
    \end{enumerate}
    \item From the structure of $B$ matrix:
    \begin{enumerate}
        \item The $NO_x$ input ($u_1$) doesn't affect any other state other than $NO_x$ concentration.

\item The urea injection dynamics ($\delta x_4$) are independent of temperature
and $NO_x$ input ($u_1$). They only depend on the injection rate ($\delta u_2$)
and the flow rate ($\delta F$)

\item The catalyst's ammonia storage only depends on temperature change and none others.
    \end{enumerate}
\end{enumerate}



\subsubsection{Output Equations}
The sensor used for $NO_x$ measurement is sensitive to the ammonia in the
tailpipe. This cross-sensitivity can be incorporated into the model by writing
the output equation as:
\begin{align*}
    y_1 &= x_1 + \chi x_2\\
\end{align*}

Thus, we have the output equation for the non-linear model:

\begin{equation}\label{eqn::ctrl_out}
    \bm{y_1 \\ y_2} = \bm{1 & \chi & 0 & 0 \\
                                 0 & 1       & 0 & 0}
                            \bm{x_1 \\ x_2 \\ x_3 \\ x_4} +
                            \bm{0 & 0 \\
                                0 & 0 \\}
                            \bm{u_1 \\ u_2}
\end{equation}

Consequently, the output equation for the linearized model:

\begin{equation}\label{eqn::lin_out}
    \bm{\delta y_1 \\ \delta y_2} = \bm{1 & \chi & 0 & 0 \\
                                 0 & 1       & 0 & 0}
                            \bm{\delta x_1 \\ \delta x_2 \\ \delta x_3 \\ \delta x_4} +
                            \bm{0 & 0 & 0 & 0\\
                                0 & 0 & 0 & 0
                                }
                            \bm{\delta u_1 \\ \delta u_2 \\ \delta T \\ \delta F}
\end{equation}
