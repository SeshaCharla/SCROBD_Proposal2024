\subsubsection{SCR-ASC Reactions}
Eley Rideal reaction mechanism \cite{hsieh2011development}
\cite{yuan2015diesel}, \cite{nova2014urea} is considered for the interpreting
the SCR reactions. The mechanism involves the following reactions:

\begin{align}
    NH_2 - CO - NH_2 (liquid) &\longrightarrow NH_2 - CO - NH_2^* + x H_2 O
                & &[\text{AdBlue evaporation}] \label{eqn::urea_1} \\
    NH_2 - CO - NH_2^*  &\longrightarrow  HNCO + NH_3
                & &[\text{Urea decomposition}] \label{eqn::urea_2}\\
    HNCO + H_2O &\longrightarrow NH_3 + CO_2
                & &[\text{Isocynic acid hydrolysis}] \label{eqn::urea_3}\\
    %===
    NH_3 + \theta_{free} &\longleftrightarrow NH_3(ads)
                & &[\text{Adsorption/Desorption}] \label{eqn::ads}\\
    %===
    4 NH_3 (ads) + 4 NO + O_2 &\longrightarrow 4 N_2 + 6 H_2O
                              & &[\text{Standard SCR reaction}]
                              \label{eqn::std_scr}\\
    %===
    2 NH_3 (ads) +  NO + N O_2 &\longrightarrow 2 N_2 + 3 H_2O
                              & &[\text{Fast SCR reaction}]
                              \label{eqn::fast_scr}\\
    %===
    4 NH_3 (ads) + 3N O_2 &\longrightarrow 3.5 N_2 + 6 H_2O
                              & &[\text{Slow SCR reaction}]
                              \label{eqn::slow_scr}\\
    %===
    4 NH_3 + 3 O_2 &\longrightarrow 2 N_2 + 6 H_2O
                         & &[\text{AMOX with/without ASC}]
                         \label{eqn::amox_N2}\\
    4 NH_3 + 5 O_2 &\longrightarrow 4 NO + 6 H_2 O
                         & &[\text{AMOX with/without ASC}]
                         \label{eqn::amox_NO}\\
    2 NH_3 + 2 O_2 &\longrightarrow N_2O + 3 H_2O
                         & &[\text{AMOX with/without ASC}]
                         \label{eqn::amox_N20}\\
    %==
    2 NO + O_2 &\longrightarrow 2 NO_2
                        & &[\text{NO oxidation}]
                        \label{eqn::NOX}
\end{align}


The PDE model for SCR reaction kinematics \cite{nova2014urea} is reduced to ODE model \cite{devarakonda2008adequacy} using by assuming "continuous stirred tank reactor (CSTR)" model (Control volume approach).


% ==============================================================================
\subsubsection{Reaction Rates}
\itbf{Assumptions:}
\begin{enumerate}
\item The reaction rates are only functions of gas-phase concentrations of $NO$,
$NH_3$, the adsorbed Ammonia and the available adsorption sites.

\item The concentration rates are converted into molar-rate so that the
mass balance in control volume approach (CSTR) can be performed. For gaseous reactants:
$$ M_g = C_g V \implies R_i = V r_i $$
The number of moles of the adsorbant is directly considered instead of their
surface concentrations.

\item A lower order Tayler approximation is assumed to model the temperature
effects in rate constant.
\end{enumerate}

\begin{enumerate}
\item Standard SCR Reaction (\ref{eqn::std_scr}):
\begin{align*}
    R_1 &= k_1 V C_{NO} M_{NH_3} = k_1V C_{NO} \Theta \theta\\
    k_1 &= A_1 e^{\frac{-E_1}{RT}}
\end{align*}

\item Ammonia Oxidation (\ref{eqn::amox_N2}):
\begin{align*}
    R_3 &= k_3 M_{NH_3} = k_3 \Theta \theta\\
    k_3 &= A_3 e^{\frac{-E_3}{RT}}
\end{align*}

\item Ammonia Adsorption/Desorption (\ref{eqn::ads}):
\begin{enumerate}
\item Forward:
\begin{align*}
    R_{4F} &= k_{4F} V C_{NH_3} \lr{\Theta - M_{NH_3}}
            = k_{4F} V C_{NH_3} \Theta \lr{1 - \theta}\\
    k_{4F} &= A_{4F} e^{\frac{-E_{4F}}{RT}}
\end{align*}

\item Reverse:
\begin{align*}
    R_{4R} &= k_{4R} M_{NH_3}
            = k_{4R} \Theta \theta \\
    k_{4R} &= A_{4R} e^{\frac{-E_{4R}}{RT}}
\end{align*}
\end{enumerate}
\end{enumerate}

Where,
\begin{align*}
    \theta &- NH_3 \text{ storage capacity fraction in SCR } = \frac{\text{Moles of $NH_3$ adsorbed} (M_{NH_3})}{\text{Total moles of $NH_3$ that can be adsorbed} (\Theta)}\\
    \Theta &- \text{Ammonia storage capacity} (moles)\\
    \Theta &= S_1 e^{S_2 T} \qquad \qquad \begin{matrix*}[l]
                S_1, S_2 &-& \text{Aging parameters of the catalyst (positve constants)}
            \end{matrix*}\\
    E_i &- \text{Activation Energy of $i^{th}$ reaction}\\
    k_i &- \text{Pre-exponential factor}\\
    R &- \text{Universal gas constant}\\
    T &- \text{Temperature}\\
    C_{\{\bullet\}} &- \text{Concentration} \lr{mol/m^3}\\
    V &- \text{Volume of the exhaust gas in the substrate\cite{devarakonda2009model}} \lr{m^3}\\
    V_e &= \epsilon A_c L_{cat}\\
        &\begin{matrix*}[l]
        A_c &-& \text{Open frontal area of the catalyst}\\
        L_{cat} &-& \text{Length of the catalyst}\\
        \epsilon &-& \text{Void fraction}
        \end{matrix*}\\
    A_c &- \text{Area of the catalyst}
\end{align*}
