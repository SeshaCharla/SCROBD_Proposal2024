\subsubsection{Non-linear Parametric Model}
% ==============================================================================
The actual input to the system is urea [from say, AdBlue ($32.5\%$ aqueous urea
solution)] injection that converted to ammonia. This can be modelled by the following equation \cite{nova2014urea}:
\begin{align*}
    \dot C_{NH_3, in} &= - \frac{1}{\tau} C_{NH_3, in} + 2 \frac{1}{\tau} \frac{ \eta u_{inj}}{N_{urea} F}\\
    \text{where, } \quad &\\
    \tau &- \text{Time constant}\\
    u_{inj} &- \text{Mass injection rate of the AdBlue solution}\\
    \eta &- \text{Mass fraction of urea in the solution}\\
    N_{urea} &- \text{Atomic number of urea}\\
    F &- \text{Exhaust flow rate of the catalyst } m^3/s
\end{align*}

\itbf{Assumptions}:
\begin{enumerate}
    \item The above model assumes that the evaporation of the urea-solutions is a significantly slower process as
        compared to it's decomposition into ammonia. Thus, the reaction
        kinetics are neglected, and the evaporation is considered are a first
        order process w.r.t the vapor pressure of the ammonia.
    \item The injection dynamics are completely decoupled from that of other
        states.
    \item Further, it is observed that Urea is completely converted to Ammonia
        at the very upstream part of the SCR catalyst
        \cite{hsieh2011development}.
\end{enumerate}

Reparametrizing the above equation, let,
\begin{align*}
    x_4 &= C_{NH_3, in} \qquad b_u = 2 \frac{ \eta}{N_{urea}} \qquad \omega_u = \frac{1}{\tau}\\
    u_2 &= u_{inj}
\end{align*}

\begin{equation}{\label{eqn::urea_inj}}
    \dot x_4 = - \omega_u x_4 +   \frac{\omega_u b_u}{F} u_{2}
\end{equation}

Using the parameetric model with $x_3 = M_{NH_3}$ and introducing urea dosing
dyanmics from eqn.~\ref{eqn::urea_inj}. ($u_2$ becomes $x_4$.)
\begin{align*}
    \bm{x_1 \\ x_2 \\ x_3 \\ x_4} = \bm{C_{NO} \\ C_{NH_3} \\ M_{NH_3} \\ C_{NH_3, in}} \qquad &
    \bm{u_1 \\ u_2 } = \bm{C_{NO, in} \\ u_{inj}}
\end{align*}
\begin{align*}
    \mat{
    \\f_{13} &=& k_1
    \\f_{23} &=& k_{4F}
    \\f_{32} &=& k_{4F} \Theta
    \\f_{31} &=& k_1 V
    \\f_{24} &=& b_v F
    }
    \qquad
    \mat{
    \\ g_1    &=& b_v F
    \\ g_2    &=& b_v F + k_{4F} \Theta
    \\ g_{3}  &=& k_{4R}+k_3
    \\ g_{23} &=& k_{4R} V^{-1}
    \\ g_{32} &=& k_{4F} V \Theta
    \\ g_4 &=& \omega_u
    }
    \qquad
    \mat{
        b_{11} &=& b_v F
        \\
        b_{42} &=& \frac{\omega_u b_u}{F}
    }
\end{align*}
\begin{equation}{\label{eqn::full_nonlinear}}
     \bm{\dot x_1 \\
        \dot x_2\\
        \dot x_3\\
        \dot x_4} =
    \bm{
        -f_{13} x_1 x_3
        -g_1 x_1
        \\
        %===
        -g_2 x_2
        + f_{23} x_2 x_3
        + g_{23} x_3
        + f_{24} x_4
        \\
        %===
        -f_{32} x_2 x_3
        -g_3 x_3
        -f_{31} x_1 x_3
        + g_{32} x_2
        \\
        %===
        -g_4 x_4
    }
    + \bm{b_{11} & 0\\
          0     & 0\\
          0     & 0\\
          0     & b_{42}  }\bm{u_1 \\ u_2 }
\end{equation}
