\section{SCR-ASC Reactions and Dynamics}

\subsection{SCR-ASC Reactions}
Eley Rideal reaction mechanism \cite{hsieh2011development}
\cite{yuan2015diesel}, \cite{nova2014urea} is considered for the interpreting
the SCR reactions. The mechanism involves the following reactions:

\begin{align}
    NH_2 - CO - NH_2 (liquid) &\longrightarrow NH_2 - CO - NH_2^* + x H_2 O
                & &[\text{AdBlue evaporation}] \label{eqn::urea_1} \\
    NH_2 - CO - NH_2^*  &\longrightarrow  HNCO + NH_3
                & &[\text{Urea decomposition}] \label{eqn::urea_2}\\
    HNCO + H_2O &\longrightarrow NH_3 + CO_2
                & &[\text{Isocynic acid hydrolysis}] \label{eqn::urea_3}\\
    %===
    NH_3 + \theta_{free} &\longleftrightarrow NH_3(ads)
                & &[\text{Adsorption/Desorption}] \label{eqn::ads}\\
    %===
    4 NH_3 (ads) + 4 NO + O_2 &\longrightarrow 4 N_2 + 6 H_2O
                              & &[\text{Standard SCR reaction}]
                              \label{eqn::std_scr}\\
    %===
    2 NH_3 (ads) +  NO + N O_2 &\longrightarrow 2 N_2 + 3 H_2O
                              & &[\text{Fast SCR reaction}]
                              \label{eqn::fast_scr}\\
    %===
    4 NH_3 (ads) + 3N O_2 &\longrightarrow 3.5 N_2 + 6 H_2O
                              & &[\text{Slow SCR reaction}]
                              \label{eqn::slow_scr}\\
    %===
    4 NH_3 + 3 O_2 &\longrightarrow 2 N_2 + 6 H_2O
                         & &[\text{AMOX with/without ASC}]
                         \label{eqn::amox_N2}\\
    4 NH_3 + 5 O_2 &\longrightarrow 4 NO + 6 H_2 O
                         & &[\text{AMOX with/without ASC}]
                         \label{eqn::amox_NO}\\
    2 NH_3 + 2 O_2 &\longrightarrow N_2O + 3 H_2O
                         & &[\text{AMOX with/without ASC}]
                         \label{eqn::amox_N20}\\
    %==
    2 NO + O_2 &\longrightarrow 2 NO_2
                        & &[\text{NO oxidation}]
                        \label{eqn::NOX}
\end{align}


The PDE model for SCR reaction kinematics \cite{nova2014urea} is reduced to ODE model \cite{devarakonda2008adequacy} using by assuming "continuous stirred tank reactor (CSTR)" model (Control volume approach).


\subsection{General reaction kinetics}
For a stoichiometric reaction of the following form:
\begin{align*}
    a A + b B \longrightarrow c C + d D
\end{align*}
We have the reaction rate \cite{chem_kine}:
\begin{align*}
    r &= -\frac{1}{a} \frac{d [A]}{dt}
       = -\frac{1}{b} \frac{d [B]}{dt}
       = \frac{1}{c}  \frac{d [C]}{dt}
       = \frac{1}{d}  \frac{d [D]}{dt}\\
    r &= k [A]^m [B]^n\\
    \text{Where, }\quad &\\
    [\bullet] &- \text{Concentration of the reactant } \bullet\\
    k &- \text{Rate constant}\\
    m, n &- \text{Constant exponents, ($m+n$) is the order of reaction}\\
\end{align*}
In the subsequent derivations, the exponents in the rate equations are limited
to either zero or 1 $m, n \in \{0, 1\}$ (This is consistent with the assumption
that reaction rates are proportional to gas-phase concentrations from \cite{devarakonda2009model}).

The rate constant can be determined using \itbf{Arrhenius equation}:
\begin{align*}
    k &= A e^{E_a/RT}\\
    \text{Where,} \quad &\\
    A &- \text{Pre-exponential factor}\\
    E_a &- \text{Activation energy}\\
    T &- \text{Temperature}\\
    R &- \text{Universal gas constant}
\end{align*}
\subsubsection{Effects of temperature change on rate constant}
We have the Arrhenius equation:
\begin{align*}
    k &= A e^{-E_a/RT}\\
    \implies \delta k &= \delta T A \lr{\frac{E_a}{RT^2}} e^{-E_a/RT} = \delta T k \lr{\frac{E_a}{RT^2}}\\
\end{align*}
In general, $E_a$ is several orders of magnitude greater than the other
parameters. Thus, small change in temperature gets amplified as the change in
the rate-constant.

Also, we have the first-order tayer approximation for the rate-constant
variation with temperature:
\begin{align*}
    k(T_0 + \delta T) &\approx k(T_0) + \frac{E_a}{RT_0^2} k(T_0) \delta T\\
    k(\delta T) &\approx k_0 + p_0k_0 \delta T
\end{align*}
This form would be more amenable to parameter estimation despite introducing
approximation errors.



%===============================================================================
\subsection{Eley-Rideal Mechanism}
In this mechanism, proposed in 1938 by D. D. Eley and E. K. Rideal, only one of
the molecules adsorbs and the other one reacts with it directly from the gas
phase, without adsorbing ("non-thermal surface reaction") \cite{eley_rideal}.
\begin{align*}
    A(g) + S(s) &\rightleftharpoons AS(s)  \qquad (k_1, k_{-1})\\
    AS(s) + B(g) &\longrightarrow Products \qquad (k)
\end{align*}

We have the rate of the second reaction:
\begin{align*}
    r &= k C_s \theta C_B\\
    \theta &= \frac{C_{AS}}{C_S}
\end{align*}
Thus,
\begin{align*}
    \frac{d C_{AS}}{dt} &= k_1 C_A C_S (1-\theta) - k_{-1} \theta C_S - k C_S \theta C_B\\
    \text{At equilibrium}:\qquad &\\
    \frac{d C_{AS}}{dt} &= 0
    \implies \theta = \frac{k_1 C_A C_S}{k_1 C_A C_s + k_{-1}C_S + k C_s C_B}
\end{align*}
Based on the rate of individual reactions, we have the following
simplifications:
\begin{enumerate}
    \item If the limiting step is adsorption/desorption:
    \begin{align*}
        \implies k C_B &\gg k_1 C_A, k_{-1}\\
        \implies \theta &= 1 \implies r = k C_S C_B
    \end{align*}

    \item If the limiting step is the reaction:
    \begin{align*}
        \implies k C_B &\ll k_1 C_A, k_{-1}\\
        \implies \theta &= \frac{k_1 C_A C_S}{k_1 C_A C_S + k_{-1}C_S}
                        = \frac{K_{ad} C_A}{K_{ad}C_A + 1} \qquad
        \text{where, } K_{ad} = \frac{k_1}{k_{-1}} \qquad
        [\text{Langmuir's Isotherm}]\\
        \implies r &= k C_S C_B \frac{K_{ad} C_A}{K_{ad}C_A + 1}
    \end{align*}
The above expression can be further simplified based on the relative
concentrations of A and B:
\begin{enumerate}
    \item High concentrations of A, $(C_A \gg C_B)$:
    \begin{align*}
        \implies r &= k C_S C_B
    \end{align*}
    The order of the reaction is zero w.r.t A.
    \item Low concentrations of A, $(C_A \ll C_B)$:
    \begin{align*}
        r &= k C_S K_{ad} C_A C_B = k_r C_S C_A C_B
    \end{align*}
\end{enumerate}
\end{enumerate}

\itbf{Note}: Arrhenius equation can be used to model the temperature dependence
of the adsorption and desorption rate constants (thermal-desroption).

% ==============================================================================

% ==============================================================================
\subsection{Reaction Rates}
\itbf{Assumptions:}
\begin{enumerate}
\item The reaction rates are only functions of gas-phase concentrations of $NO$,
$NH_3$, the adsorbed Ammonia and the available adsorption sites.

\item The concentration rates are converted into molar-rate so that the
mass balance in control volume approach (CSTR) can be performed. For gaseous reactants:
$$ M_g = C_g V \implies R_i = V r_i $$
The number of moles of the adsorbant is directly considered instead of their
surface concentrations.

\item A lower order Tayler approximation is assumed to model the temperature
effects in rate constant.
\end{enumerate}

\begin{enumerate}
\item Standard SCR Reaction (\ref{eqn::std_scr}):
\begin{align*}
    R_1 &= k_1 V C_{NO} M_{NH_3} = k_1V C_{NO} \Theta \theta\\
    k_1 &= A_1 e^{\frac{-E_1}{RT}}
\end{align*}

\item Ammonia Oxidation (\ref{eqn::amox_N2}):
\begin{align*}
    R_3 &= k_3 M_{NH_3} = k_3 \Theta \theta\\
    k_3 &= A_3 e^{\frac{-E_3}{RT}}
\end{align*}

\item Ammonia Adsorption/Desorption (\ref{eqn::ads}):
\begin{enumerate}
\item Forward:
\begin{align*}
    R_{4F} &= k_{4F} V C_{NH_3} \lr{\Theta - M_{NH_3}}
            = k_{4F} V C_{NH_3} \Theta \lr{1 - \theta}\\
    k_{4F} &= A_{4F} e^{\frac{-E_{4F}}{RT}}
\end{align*}

\item Reverse:
\begin{align*}
    R_{4R} &= k_{4R} M_{NH_3}
            = k_{4R} \Theta \theta \\
    k_{4R} &= A_{4R} e^{\frac{-E_{4R}}{RT}}
\end{align*}
\end{enumerate}
\end{enumerate}

Where,
\begin{align*}
    \theta &- NH_3 \text{ storage capacity fraction in SCR } = \frac{\text{Moles of $NH_3$ adsorbed} (M_{NH_3})}{\text{Total moles of $NH_3$ that can be adsorbed} (\Theta)}\\
    \Theta &- \text{Ammonia storage capacity} (moles)\\
    \Theta &= S_1 e^{S_2 T} \qquad \qquad \begin{matrix*}[l]
                S_1, S_2 &-& \text{Aging parameters of the catalyst (positve constants)}
            \end{matrix*}\\
    E_i &- \text{Activation Energy of $i^{th}$ reaction}\\
    k_i &- \text{Pre-exponential factor}\\
    R &- \text{Universal gas constant}\\
    T &- \text{Temperature}\\
    C_{\{\bullet\}} &- \text{Concentration} \lr{mol/m^3}\\
    V &- \text{Volume of the exhaust gas in the substrate\cite{devarakonda2009model}} \lr{m^3}\\
    V_e &= \epsilon A_c L_{cat}\\
        &\begin{matrix*}[l]
        A_c &-& \text{Open frontal area of the catalyst}\\
        L_{cat} &-& \text{Length of the catalyst}\\
        \epsilon &-& \text{Void fraction}
        \end{matrix*}\\
    A_c &- \text{Area of the catalyst}
\end{align*}
